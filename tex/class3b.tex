%% This document created by Scientific Word (R) Version 2.0


\documentclass[thmsa]{article}
%%%%%%%%%%%%%%%%%%%%%%%%%%%%
\usepackage{sw20lart}

%TCIDATA{TCIstyle=Article/art4.lat,lart,article}

\input tcilatex
\QQQ{Language}{
American English
}

\begin{document}

\title{Classical Trajectories in Coulomb Three Body Systems\\
Trayectorias Cl\'asicas en Sistemas Coulombianos de Tres Cuerpos}
\author{Fernando P\'erez \\
%EndAName
Departamento de F\'\i sica\\
Universidad de Antioquia\\
A.A. 1226\\
Medell\'\i n, Colombia\\
e-mail: fperez@uniandes.edu.co \and Jorge Mahecha \\
%EndAName
Departamento de F\'\i sica\\
Universidad de Antioquia\\
A.A. 1226\\
Medell\'\i n, Colombia\\
e-mail: mahecha@fisica.udea.edu.co}
\date{}
\maketitle

\begin{abstract}
A computer program for calculating orbits and Poincar\'e sections in the
classical Coulomb Three Body Problem with arbitrary masses and charges in
three dimensional space is presented. The regularization procedure necessary
to obtain a reliable numerical integration of the equations of motion is
sketched. Different configurations of the Helium atom and the Positronium
negative ion ($Ps^{-}$), obtained with the program, are analyzed.

\medskip\ 

\begin{center}
\textbf{Resumen}
\end{center}

Se presenta un programa de computador para calcular \'orbitas y secciones de
Poincar\'e en el Problema de los Tres Cuerpos Coulombiano cl\'asico, con
masas y cargas arbitrarias. Se esboza el procedimiento de regularizaci\'on
necesario para obtener una integraci\'on num\'erica confiable de las
ecuaciones de movimiento. Se analizan diferentes configuraciones obtenidas
con el programa para el \'atomo de Helio y el ion negativo de Positronio ($%
Ps^{-}$).

PACS: 31.15.+q; 31.10.+z; 36.10.Dr.
\end{abstract}

\section{Introduction}

The Three Body Problem (TBP), stated as the study of the evolution of three
point-like particles whose interaction is gravitational or electrostatic in
an otherwise empty space, is a long time classic and perhaps one of the most
frequently tackled challenges of mathematical physics.

However, the non-integrable character of the problem renders useless the
attempt of finding a function of time describing the three particles'
evolution after starting from arbitrary initial conditions. This forces the
adoption of other strategies, mainly:

\begin{enumerate}
\item[(a)]  A qualitative approach where questions of topological character
concerning the solutions are addressed.

\item[(b)]  Attempts to obtain analytical solutions to cases in which
certain symmetries (geometrical, dynamical or both) permit simplifications
to be performed.

\item[(c)]  Numerical solutions to cases ranging from restricted ones such
as those mentioned in (b) to the most general one where all parameters can
be varied.
\end{enumerate}

In this work, we shall present a computer program developed in the language
C++, which calculates orbits for the electrostatic TBP with arbitrary
masses, charges, and initial conditions in three dimensional space.

After introducing the geometry and notation of our problem, we shall sketch
the necessary mathematical tools for integrating the equations of motion of
the TBP, in a fashion that reliably handles the singularities associated to
the Coulomb potential in binary collisions. We will then present some
details concerning the program itself, and will finish with some results
obtained with it on the helium atom and the positronium negative ion ($%
Ps^{-} $)\allowbreak \allowbreak .

At this point it should perhaps be noted that, even though these are systems
which in rigor should be treated quantum mechanically, theoretical
developments from the last few decades have shown the relevance of a
detailed knowledge of the \emph{classical} behavior of non-integrable
systems in the analysis of many purely quantum aspects (see esp. \cite
{Gutzwiller}, but also \cite{Delande}, \cite{Nauenberg}).

\section{The three body Hamiltonian}

Let us then consider a system consisting of three point particles with
masses $m_i$ and charges $Z_i$ ($i=1,2,3$) interacting through a Coulomb
potential in free space. The coordinates and momenta of particle 1 will be
described by the vectors $\mathbf{q}_1^{\prime }=\left( q_1^{\prime
},q_2^{\prime },q_3^{\prime }\right) $ and $\mathbf{p}_1^{\prime }=\left(
p_1^{\prime },p_2^{\prime },p_3^{\prime }\right) $ respectively, with
analogous expressions ---adjusting the indices accordingly--- for the other
two. The geometry of the problem is illustrated in Fig. 1, where the
inter-particle distances $R$, $R_1$ and $R_2$ are indicated.

Since a system with all three charges of equal sign is of no physical
interest when looking for bound states, we assume without loss of generality
that particles 1 and 2 have charges $-Z_1$ and $-Z_2$, while particle 3 has
charge $+Z_3$, with $Z_i>0$ for $i=1,2,3$. In atomic units, and considering
only Coulomb interactions, the Hamiltonian for this system is 
\begin{equation}
H^{\prime }=\sum_{r=1}^3\frac{\mathbf{p}_r^{\prime 2}}{2m_r}-\frac{Z_1Z_3}{%
R_1}-\frac{Z_2Z_3}{R_2}+\frac{Z_1Z_2}R.  \label{e1}
\end{equation}

The associated canonical equations of motion read 
\begin{equation}
\frac{dq_r^{\prime }}{dt}=\frac{\partial H^{\prime }}{\partial p_r^{\prime }}%
,\qquad \frac{dp_r^{\prime }}{dt}=-\frac{\partial H^{\prime }}{\partial
q_r^{\prime }}\qquad (r=1,2,\ldots ,9).  \label{e2}
\end{equation}

It is quite clear that the Hamiltonian from Eq. (\ref{e1}) is singular
whenever any of the interparticle distances ($R$, $R_1$ or $R_2$) vanishes.
From a strictly numerical standpoint, we will have problems when any of
those distances becomes very small, even if it does not strictly vanish,
since in Eq. (\ref{e1}) terms with very different magnitudes will have to be
added.

\section{Regularization procedure}

One standard way of avoiding the problems associated to binary collisions in
the Hamiltonian (\ref{e1}) ---the three-particle collision cannot be
regularized--- was developed by Aarseth and Zare \cite{Aarseth}, based on
previous work by Kustaanheimo and Steifel \cite{Kust}. We will sketch here
this procedure, whose detailed description can be found in \cite{Aarseth},
and more concisely in \cite{Richter(clasicos)}.

The basic idea is to obtain a canonical transformation which gives the
Hamiltonian a non-singular form in the case of binary collisions, albeit at
the expense of producing a set of dynamical equations significantly more
complex in their algebraic expression. The resulting set of coordinates and
momenta will be called regularized coordinates $(Q,P)$.

\subsection{Relative coordinates and extension of phase space}

Briefly, in order to go from the $(q^{\prime },p^{\prime })$ to the $(Q,P)$
we first change to variables relative to $m_3$ with 
\begin{equation}
\begin{array}{lc}
\mathbf{q}_k=\mathbf{q}_k^{\prime }-\mathbf{q}_3^{\prime }\quad \quad & 
(k=1,2) \\ 
\mathbf{p}_k=\mathbf{p}_k^{\prime }\quad \quad & (k=1,2) \\ 
\mathbf{q}_3=\mathbf{q}_3^{\prime } &  \\ 
\mathbf{p}_3=\mathbf{p}_1^{\prime }+\mathbf{p}_2^{\prime }+\mathbf{p}%
_3^{\prime }. & 
\end{array}
\label{e3}
\end{equation}

We see from Eq. (\ref{e3}) that $\mathbf{p}_3$ is the total momentum of the
system. We may then, without loss of generality, assume that in the original
system $(q^{\prime },p^{\prime })$ the center of mass is at rest, and we
take it to be, for the sake of algebraic convenience, located at the origin.
This eliminates six variables from our equations, and leaves us with a
two-particle problem, which is however \emph{not} further reducible to an
equivalent one-particle description.

We next expand the dimension of the phase space by renaming variables
according to 
\begin{equation}
q_r\rightarrow q_{r+1},\quad \quad p_r\rightarrow p_{r+1}\quad \quad
(r=4,5,6),  \label{e4}
\end{equation}
leaving the variables with $r\leq 3$ as before and defining the mock
coordinates and momenta 
\begin{equation}
q_4\equiv 0,\quad \quad q_8\equiv 0,\quad \quad p_4\equiv 0,\quad \quad
p_8\equiv 0.  \label{e5}
\end{equation}

A discussion regarding the need for this extension can be found in \cite
{Aarseth}.

\subsection{Regularized coordinates and new time}

We now introduce the regularized coordinates through a canonical
transformation of the $(\mathbf{q},\mathbf{p})$ to the $(\mathbf{Q},\mathbf{P%
})$ associated to a generating function 
\begin{equation}
W=W(\mathbf{p},\mathbf{Q})=\sum_{r=1}^8p_rf_r(\mathbf{Q}),  \label{e5b}
\end{equation}
using the notation 
\begin{equation}
\mathbf{p}=(p_1,p_2,p_3,p_4,p_5,p_6,p_7,p_8),\mathbf{Q}%
=(Q_1,Q_2,Q_3,Q_4,Q_5,Q_6,Q_7,Q_8).
\end{equation}

The $f_r(\mathbf{Q})$ are continuous, differentiable functions yet
unspecified, but which must obey the conditions 
\begin{eqnarray}
f_r &=&f_r(Q_1,Q_2,Q_3,Q_4)\quad \quad (r\leq 4),  \label{e5c} \\
f_r &=&f_r(Q_5,Q_6,Q_7,Q_8)\quad \quad (r>4).  \label{e5cc}
\end{eqnarray}

Let $\mathbf{A}$ be the matrix with components 
\begin{equation}
A_{rs}=\frac{\partial f_s}{\partial Q_r},
\end{equation}
and $\mathbf{A}_1,$ $\mathbf{A}_2$ its two non-zero blocks located on the
diagonal.

If we assume the $f_r(\mathbf{Q})$ to be such that 
\begin{equation}
\mathbf{A}_k\mathbf{A}_k^T=\lambda _kR_k\mathbf{I}\qquad (k=1,2)  \label{e5d}
\end{equation}
where the $\lambda _k$ are real numbers, $\mathbf{I}$ is the $4\times 4$
identity matrix and $T$ represents matrix transposition, we can write our
Hamiltonian as 
\begin{equation}
\widetilde{H}\left( \mathbf{Q},\mathbf{P}\right) =\frac{\mathbf{P}_1^2}{2\mu
_{13}\lambda _1R_1}+\frac{\mathbf{P}_2^2}{2\mu _{23}\lambda _2R_2}+\frac{%
\mathbf{P}_1^T\mathbf{A}_1\mathbf{A}_2^T\mathbf{P}_2}{m_3\lambda _1\lambda
_2R_1R_2}-\frac{Z_1Z_3}{R_1}-\frac{Z_2Z_3}{R_2}+\frac{Z_1Z_2}R,  \label{e6a}
\end{equation}
where 
\begin{equation}
\mathbf{P}_1=(P_1,P_2,P_3,P_4),\quad \quad \mathbf{P}_2=(P_5,P_6,P_7,P_8).
\end{equation}

Now, if in a system described by a time independent Hamiltonian $H$ with
parameter \emph{$t,$} we include \emph{$t$} as a coordinate and introduce a
new time parameter $\tau $, the new Hamiltonian $H^{*}$ will be given by 
\begin{equation}
H^{*}=\left( H-E\right) \frac{dt}{d\tau },
\end{equation}
being $E$ the total energy of the system.

Therefore, by taking 
\begin{equation}
dt=R_1R_2d\tau
\end{equation}
we may avoid the divergences in Eq. (\ref{e6a}) associated to $R_i\approx
0,\,i=1,2$.

In order for the regularization procedure to actually work we must now
provide a set $\left\{ f_r(\mathbf{Q})\right\} $ of functions which
satisfies the requested conditions, Eqs. (\ref{e5c}), (\ref{e5cc}) and (\ref
{e5d}) . One such set is given by the Kustaanheimo-Steifel transformations,
defined as 
\begin{equation}
\begin{array}{ll}
f_1=Q_1^2-Q_2^2-Q_3^2+Q_4^2,\quad \quad & f_5=Q_5^2-Q_6^2-Q_7^2+Q_8^2, \\ 
f_2=2\,(Q_1Q_2-\,Q_3Q_4),\quad \quad & f_6=2(Q_5Q_6-\,Q_7Q_8), \\ 
f_3=2\,(Q_1Q_3+\,Q_2Q_4),\quad \quad & f_7=2(Q_5Q_7+Q_6Q_8), \\ 
f_4=0,\quad \quad & f_8=0,
\end{array}
\end{equation}
which lead to condition (\ref{e5d}) being valid with $\lambda _1=\lambda
_2=4 $. The resulting regularized Hamiltonian is finally given by 
\begin{equation}
\begin{array}{r}
\Gamma (\mathbf{Q},\mathbf{P})=\dfrac{R_2}{8\mu _{13}}\mathbf{P}_1^2+\dfrac{%
R_1}{8\mu _{23}}\mathbf{P}_2^2+\dfrac 1{16m_3}\mathbf{P}_1^T\mathbf{A}_1%
\mathbf{A}_2^T\mathbf{P}_2 \\ 
-Z_3\left( R_1Z_2+R_2Z_1\right) +R_1R_2\left( \dfrac{Z_1Z_2}R-E\right) ,
\end{array}
\label{f7}
\end{equation}
where the reduced masses with respect to $m_3$ are 
\begin{equation}
\mu _{13}=\frac{m_1m_3}{m_1+m_3},\qquad \mu _{23}=\frac{m_2m_3}{m_2+m_3}.
\label{f8}
\end{equation}

The canonical equations of motion associated to Eq. (\ref{f7}) are then 
\begin{equation}
\frac{dQ_r}{d\tau }=\frac{\partial \Gamma }{\partial P_r},\qquad \frac{dP_r}{%
d\tau }=-\frac{\partial \Gamma }{\partial Q_r}\qquad (r=1,2,\ldots ,8).
\label{f9}
\end{equation}

The explicit form of the above equations for the general three dimensional
case is quite cumbersome, although easily obtained with a symbolic
manipulation program (such as Maple or Mathematica). It is nonetheless
useful to note that the derivatives $\partial R/\partial Q_i$ can be
expressed in a compact form which is more efficient for numerical purposes
than the one obtained by direct differentiation. By defining the $8\times 1$
column vector $\mathbf{g}$ as 
\begin{equation}
\mathbf{g}=\left[ 
\begin{array}{c}
\mathbf{f}_1-\mathbf{f}_2 \\ 
\mathbf{f}_2-\mathbf{f}_1
\end{array}
\right]  \label{e18}
\end{equation}
where the vectors $\mathbf{f}_1$ and $\mathbf{f}_2$ are 
\begin{equation}
\mathbf{f}_1=\left[ 
\begin{array}{c}
Q_1^2-Q_2^2-Q_3^2+Q_4^2 \\ 
2\,(Q_1Q_2-\,Q_3Q_4) \\ 
2\,(Q_1Q_3+\,Q_2Q_4) \\ 
0
\end{array}
\right] ,\quad \quad \mathbf{f}_2=\left[ 
\begin{array}{c}
Q_5^2-Q_6^2-Q_7^2+Q_8^2 \\ 
2(Q_5Q_6-\,Q_7Q_8) \\ 
2(Q_5Q_7+Q_6Q_8) \\ 
0
\end{array}
\right] ,  \label{e16}
\end{equation}
one can verify that the following relations hold \cite{Perez}: 
\begin{equation}
\frac{\partial R}{\partial Q_i}=\frac 1R(\mathbf{Ag})_i,\qquad (i=1,2,\ldots
,8).  \label{e20}
\end{equation}

\subsection{Transformation equations}

The explicit expressions for the transformations to regularized coordinates,
resulting from the above definitions, are 
\begin{equation}
\begin{array}{l}
Q_1=\left[ \frac 12(\left| \mathbf{q}_1\right| +q_1)\right] ^{1/2} \\ 
Q_2=q_2/2Q_1 \\ 
Q_3=q_3/2Q_1 \\ 
Q_4=0,
\end{array}
\label{e6}
\end{equation}
for $q_1\geq 0$ or 
\begin{equation}
\begin{array}{l}
Q_2=\left[ \frac 12(\left| \mathbf{q}_1\right| -q_1)\right] ^{1/2} \\ 
Q_1=q_2/2Q_1 \\ 
Q_3=0 \\ 
Q_4=q_3/2Q_2,
\end{array}
\label{e7}
\end{equation}
for $q_1<0.$ For the variables $(Q_5,Q_6,Q_7,Q_8)$ we have analogous
expressions, obtained by adding 4 to all indices and replacing $\mathbf{q}_1$
by $\mathbf{q}_2$.

The interparticle distances $R_1,\,R_2$ and $R$, expressed in terms of the
regularized coordinates, can be cast in the form 
\begin{equation}
R_1=\sum_{r=1}^4Q_r^2,\quad \quad R_2=\sum_{r=5}^8Q_r^2,  \label{e14}
\end{equation}
and 
\begin{equation}
R^2=R_1^2+R_2^2-2\mathbf{f}_1\cdot \mathbf{f}_2.  \label{e15}
\end{equation}

Finally, the regularized momenta are given by 
\begin{equation}
\mathbf{P}_k=\mathbf{A}_k\mathbf{p}_k\quad \quad (k=1,2),  \label{e10}
\end{equation}
where 
\begin{equation}
\mathbf{p}_1=(p_1,p_2,p_3,p_4),\quad \quad \mathbf{p}_2=(p_5,p_6,p_7,p_8).
\label{e5a}
\end{equation}

\section{Calculation program}

Through the time scale change, the effect of the regularization is to take
very small time steps near the binary collisions ($R_1\approx 0$ or $%
R_2\approx 0$), while keeping a non-divergent Hamiltonian. However, despite
this intrinsically adaptive nature of the regularized formulation, numerical
work shows that if one wishes to balance efficiency and precision, it is
necessary to implement an adaptive step size algorithm for the differential
equation integrator.

Our program, developed in C++, uses a standard 4$^{\text{th}}$ order
Runge-Kutta adaptive step size integrator \cite{NumRecipes} driven by a
routine which takes care of coordinate transformations, memory storage, disk
swapping of data and Poincar\'e section calculations. In order to improve
the accuracy of the results, Poincar\'e sections are obtained using reverse
integration over the trajectory to find the intersection between the orbit
and the Poincar\'e surface instead of interpolation, a method proposed by M.
H\'enon \cite{Henon(Poincare)}.

The program attempts to be useful for interactive exploration of the TBP,
allowing control of the parameters which define the system and the
integration conditions such as charges, masses, positions and momenta, error
tolerance, etc. Initial conditions are defined in laboratory coordinates,
integration is performed in the regularized ones, and position output data
files are generated in center of mass coordinates for viewing convenience
(to eliminate from the plots uniform displacements of the system). Besides
position data, the program generates files with relative coordinates,
Poincar\'e sections and a time-energy file which, since the system is
conservative, serves as a checkout file for numerical problems. Langmuir
type orbits \cite{Dimitri} and Poincar\'e sections in hyperspherical
coordinates can be obtained through menu options.

Since considerable simplification of the equations of motion and coordinate
transformations results from assuming that the three particles evolve in a
plane, the source code includes all the expressions corresponding to both
the two and three dimensional cases. This allows the compilation of programs
for both types of configurations, with fast integration of the plane cases
and a slower but general option for three dimensional studies.

The resulting program, usable as a tool for numerical exploration of the
general Coulomb TBP, produces accurate orbits in reasonable times on a
typical desktop computer for most three dimensional configurations. Source
code is available from the author upon request.

Finally, we shall present some results obtained with the program for two
typical three body systems, namely the Helium atom and the Positronium
negative ion (i.e. $e^{-}e^{+}e^{-},$ abbreviated $Ps^{-}$).

\section{The Helium atom}

\subsection{Wannier orbits}

In the so-called Wannier configuration, both electrons orbit around the
nucleus with equal angular momentum, satisfying 
\begin{equation}
\mathbf{r}_1\left( t\right) =-\mathbf{r}_2\left( t\right) .
\end{equation}

It can be shown \cite{Richter(clasicos)}, that this condition leads to
elliptic trajectories. This type of orbit was proposed as one of the first
classical models for the Helium atom, but the presence of a diverging
Lyapunov exponent leads to expect it to be of little significance as a
realistic description of this system. Quantum calculations confirm this
prediction (see \cite{Richter(clasicos)} for a discussion), which can be
also checked numerically by performing slight variations in the initial
conditions. Fig. 2 (a) shows a typical Wannier orbit, and in Fig. 2 (b) we
have a horizontal displacement of $10^{-4}$ in the initial position of one
electron, leading rapidly to ionization. This suggests that the orbit is
unstable, as other types of perturbations to the initial conditions confirm.

\subsection{Langmuir type orbits}

This is another classical model for the He atom, which exhibits the shortest
periodic orbit of this system [see Fig. 3(b)], an orbit surrounded by a
stability island in phase space. Although strictly speaking the Langmuir
orbit is the one just mentioned, one can study the family of orbits whose
initial conditions exhibit the symmetry shown in Fig. 3(a). Our program can
generate orbits of this type asking for the values of $a,$ $b$ and the total
energy $E$: those shown in Figs. 3(b) and 3(c) are obtained with $a=1.40706$
and $a=0.25$ respectively ($E=-1$, $b=0$ for both).

The stability of these orbits in phase space can be easily seen by making a
Poincar\'e map in terms of the hyperspherical radius $R$ defined as 
\begin{equation}
R=\sqrt{\left| \mathbf{r}_1\right| ^2+\left| \mathbf{r}_2\right| ^2}
\label{hyperrad}
\end{equation}
and its associated canonical momentum $p_R$ \cite{Mahecha(LAMP)}. Such a map
(with $E=-1$, $b=0$ for 28 orbits varying $a$ from 0.05 to 1.4) is shown in
Fig. 3(d) where the ``basic'' orbit from Fig. 3(a) appears as a dot at $%
R=1.98988$ (i.e., $a=1.40706$).

Now, numerical testing with perturbations on the initial conditions which
alter the symmetry of the configuration shows that the ``basic'' orbit is
stabler than more complex ones. Figs. 4(a) and 4(b) were respectively
obtained with $a=1.40706,\,b=0,\,E=-1,$ and $a=2,\,b=0,\,E=-1$ as starting
points, introducing then a change in the $x$-coordinate of \emph{one}
electron of magnitude $\delta x=10^{-3}$ in the first case and $\delta
x=10^{-6}$ in the second. In Fig. 4(a) no ionization occurs with an
integration time of $t=100$ (atomic units), while Fig. 4(b) shows
auto-ionization after $t=45.$

This result agrees with what can be inferred from the Poincar\'e map just
discussed, for the basic orbit is surrounded by a sharply defined torus
structure which we expect to exist in phase space even when the symmetry of
the configuration is slightly broken, while such a structure does not exist
for orbits around $a=2$ (i.e., $R=4$).

\subsection{Asymmetric top}

This is another model originally proposed by Langmuir for the He atom, where
the electrons form an equilateral triangle which rotates rigidly around the
nucleus. The trajectory is shown in Fig. 5(a), resulting from the initial
conditions $\mathbf{r}_1=\left( \sqrt{3}/2,0,1/2\right) ,\,\mathbf{r}%
_2=\left( \sqrt{3}/2,0,-1/2\right) ,\,\mathbf{p}_1=\mathbf{p}_2=\left(
0,p_0,0\right) ,$ with $p_0=1.225.$ If these exact conditions are not given
but the basic symmetry is maintained, non-ionizing orbits can be obtained,
such as the one from Fig. 5(b), which has $p_0=1.4$.

\subsection{Planetary configurations}

This name is given to geometries where on the average $R_1\gg R_2$. They
possess both experimental and theoretical interest (\cite{Richter(frozen)}, 
\cite{Richter(planet)}), and can exist in various different forms. A recent
study of these configurations ---theoretical as well as numerical--- in the
context of a He atom in which one of the electrons is replaced by an
antiproton, can be found in (\cite{Benvenuto}).

If $L=0$ ($L$ is the total angular momentum), we obtain a family of
configurations where the simplest one is the so called ``frozen planet'' 
\cite{Richter(frozen)}: one electron oscillates near the nucleus and the
other remains almost stationary further away. Slightly different conditions
which respect $L=0$ produce orbits like the one in Fig. 6(a), while for $%
L\neq 0$ both electrons can evolve in complete revolutions around the
nucleus, albeit with very different frequencies: Fig. 6(b) displays one such
trajectory.

If the electrons are allowed to interact more (by not satisfying $R_1\gg R_2$
so rigidly), this interaction can produce significant deformations on the
orbits, but they may still exist for long periods: Fig. 6(c) was obtained
with initial conditions $\mathbf{r}_1=\left( 1,0,0\right) ,\,\mathbf{r}%
_2=\left( 3,0,0\right) ,\,\mathbf{p}_1=\left( 0,-1,0\right) ,\mathbf{p}%
_2=\left( 0,0.4,0\right) $ and an integration time of $t=601$ (for a
discussion, see \cite{Yamamoto}).

Finally, three-dimensional planetary configurations are also possible, such
as the one in Fig. 6(d).

\section{The $Ps^{-}$ ion}

\subsection{Langmuir orbits}

Wannier orbits in the $Ps^{-}$ ion exist identical to those in the He atom
(symmetry maintains the $e^{+}$ static in the center), and hence will not be
shown.

Langmuir type orbits, with the same geometry from Fig. 3(a) also exist in
the $Ps^{-}$ ion, although here the $e^{+}$ oscillates in a vertical
straight line (plots are in the center of mass system). In the $Ps^{-}$, the
simplest orbit has a more complex geometry than in the case of He, as shown
in Fig. 7(a), caused by the motion of the $e^{+}$ just mentioned. Here we
also have a whole family of trajectories, whose typical aspect is that of
Fig. 7(a).

The absence of an orbit as simple as in Helium is clear in the Poincar\'e
section in hyperspherical coordinates shown in Fig. 7(c): there is a
``hole'' in the center of the map, around which the simplest orbit [Fig.
7(a)] winds itself. This map seems at first sight chaotic, but this is just
a visual effect, as can be seen by plotting only a few trajectories [Fig.
7(d)]; a chaotic behavior would lead to these few orbits densely filling the
entire region (or at least some subregions) instead of just tracing finite
sets of points.

Langmuir orbits in $Ps^{-}$ show even more sensitiveness to perturbations
which destroy the symmetry of the configuration than in Helium, as can be
easily checked numerically. Here, even the simplest configuration ionizes
rapidly if its initial conditions are not exactly symmetrical.

We must note that for three particle systems with finite masses, it is
standard \cite{Botero} to use a different definition for hyperspherical
coordinates than the one given for He. Although expression (\ref{hyperrad})
remains valid for the hyperradius $R,$ and its associated canonical momentum 
$p_R$ is constructed in the usual fashion, the vectors $\mathbf{r}_1$ and $%
\mathbf{r}_2$ are now defined as 
\begin{eqnarray}
\mathbf{r}_1 &=&\mathbf{q}_2-\mathbf{q}_1, \\
\mathbf{r}_2 &=&\sqrt{\frac{\mu _{12,3}}{\mu _{12}}}\left[ -\frac 12\left( 
\mathbf{q}_1+\mathbf{q}_2\right) +\mathbf{q}_3\right] ,
\end{eqnarray}
where 
\begin{equation}
\mu _{12,3}=\frac{\left( m_1+m_2\right) m_3}{m_1+m_2+m_3},\quad \quad \mu
_{12}=\frac{m_1m_2}{m_1+m_2},
\end{equation}
and $\mathbf{q}_1,\,\mathbf{q}_2$ and $\mathbf{q}_3$ are the position
vectors of particles 1, 2 and 3 respectively, in the center of mass system.

\subsection{Asymmetric top}

The basic definitions are in this case the same as for Helium, and we also
have non-ionizing orbits, although very sensitive to symmetry-breaking
changes in the initial conditions [see Fig. 8].

\subsection{Planetary configurations}

Through numerical search, no bound planetary configurations were found for
the $Ps^{-}$ ion, with all attempts resulting in orbits like Fig. 9(a). A
plausible hypothesis for this phenomenon is that, since in $Ps^{-}$ the
positive particle has charge +1, the outer electron sees a zero mean field
from the $e^{-}e^{+}$ pair, too weak to keep it bound. In order to test the
validity of this statement, we gave the positive particle a charge +2,
keeping all other parameters equal: the resulting orbit, shown in Fig. 9(b),
is bound.

\section{Conclusions}

A general program for the numerical integration of the classical Coulomb
three body problem was presented, which uses regularized coordinates in
order to obtain reliable results even in the case of close inter-particle
encounters. The program, written in C++, permits interactive changes to all
system parameters.

In the Helium atom, we studied numerically the Wannier, Langmuir
---oscillatory and asymmetric top--- and planetary configurations. Except
for the last one, these are all geometries with special symmetries, and we
found them in general to be extremely sensitive to changes in the initial
conditions which disrupted such symmetries.

Planetary configurations, not associated to any specific geometrical
symmetry, were found to be more robust with respect to variations in the
initial conditions. This suggests that they may occupy, with non-ionizing
orbits, a larger region in phase space than do the aforementioned
configurations. Since both periodic and quasi-periodic orbits were found,
and because of their significance in phase space, they appear as important
geometries for a semi-classical quantization of certain states of the He
atom, especially doubly excited states with appreciably different quantum
numbers for both electrons.

With the $Ps^{-}$ ion, we found results analogous to those of Helium,
concerning the sensitivity of all configurations with respect to the
symmetry in the initial conditions. However, the $Ps^{-}$ ion appears
numerically to be unstabler than Helium, with shorter self-ionization times
and no bound planetary configurations: only the strictly symmetrical models
lead to classically bound states. This can be regarded as a classical
justification for the experimentally observed instability of the $Ps^{-}$
ion \cite{Mills}.

Finally, the program was recently used to confirm through numerical
experiments, theoretical predictions concerning special geometries in
Coulomb three body systems (\cite{Santander}, \cite{Sant-Mahe}).

\section{Acknowledgments}

We would like to thank professor Manuel P\'aez for very helpful discussions
regarding computing aspects of the problem.

Support from Colciencias and CIEN, under contract No. 1115-05-012-92, is
gratefully acknowledged.

\begin{thebibliography}{99}
\bibitem{Aarseth}  S. J. Aarseth and K. Zare, Celest. Mech. \textbf{10}
(1974) 185.

\bibitem{Benvenuto}  F. Benvenuto, G. Casati and D. Shepelyansky, Phys. Rev.
A \textbf{53} (1996) 737.

\bibitem{Botero}  J. Botero, 1986. \emph{Ph.D. Thesis} (Louisiana State
University, Departament of Physics and Astronomy). Unpublished.

\bibitem{Delande}  D. Delande and A. Buchleitner, Adv. At. Mol. Opt. Phys. 
\textbf{34} (1994) 85.

\bibitem{Dimitri}  M. S. Dimitrijevi\'c and P. V. Gruji\'c, Z. Naturforsch. 
\textbf{39a} (1984) 930.

\bibitem{Gutzwiller}  M. C. Gutzwiller, 1990. \emph{Chaos in Classical and
Quantum Mechanics} (Berlin: Springer).

\bibitem{Henon(Poincare)}  M. H\'enon, Physica 5D (1982) 412.

\bibitem{Kust}  P. Kustaanheimo and E. Steifel, J. Reine Angew. Math. 
\textbf{218} (1965) 204.

\bibitem{Mahecha(LAMP)}  J. Mahecha, LAMP Series Report LAMP/94/1, ICTP
(1994).

\bibitem{Mills}  A. P. Mills, Phys. Rev. Lett. \textbf{50} (1983) 671.

\bibitem{Nauenberg}  M. Nauenberg, Comments At. Mol. Phys. \textbf{25}
(1990) 151.

\bibitem{Perez}  F. P\'erez, 1994. \emph{Undergraduate thesis} (Universidad
de Antioquia, Departamento de F\'\i sica). Unpublished.

\bibitem{NumRecipes}  W. H. Press et al. 1988. \emph{Numerical Recipes in C}
(Cambridge: Cambridge University Press).

\bibitem{Richter(frozen)}  K. Richter and D. Wintgen, Phys. Rev. Lett. 
\textbf{65} (1990) 1965.

\bibitem{Richter(planet)}  K. Richter and D. Wintgen, J. Phys. B: At. Mol.
Opt. Phys. \textbf{24} (1991) L565.

\bibitem{Richter(clasicos)}  K. Richter, G. Tanner and D. Wintgen, Phys.
Rev. A \textbf{48} (1993) 4182.

\bibitem{Santander}  A. Santander, 1995. \emph{Undergraduate thesis}
(Universidad de Antioquia, Departamento de F\'\i sica). Unpublished.

\bibitem{Sant-Mahe}  A. Santander, J. Mahecha and F. P\'erez. \emph{Rigid
rotor and fixed shape solutions in the Coulombic three body problem}, Few
Body Systems, presented for publication.

\bibitem{Yamamoto}  T. Yamamoto and K. Kaneko, Phys. Rev. Lett. \textbf{70 }%
(1993) 1928.
\end{thebibliography}

\newpage 

\pagestyle{empty}\ 

\textbf{{\large \textbf{FIGURE CAPTIONS}}}

\smallskip\ 

\textsc{Figure} 1. Geometry of the problem.

\textsc{Figure} 2. Wannier orbits in the He atom. (a) Typical orbit. (b)
Slightly perturbed trajectory.

\textsc{Figure} 3. Langmuir type orbits in the He atom. (a) Geometry of the
initial conditions. (b) Basic orbit (shortest periodic orbit in the He
atom). (c) A more complex orbit. (d) Poincar\'e map of a family of
trajectories in hyperspherical coordinates ($R-P_R$ plot).

\textsc{Figure} 4. Perturbations on Langmuir type orbits.

\textsc{Figure} 5. Asymmetric top orbits in the He atom.

\textsc{Figure} 6. Planetary configurations for the He atom. (a) Variation
on the frozen planet, $L=0$. (b) Periodic orbit with $L\neq 0$. (c) Orbit
where the electrons are closer, and therefore may have a stronger
interaction. (d) Three-dimensional configuration.

\textsc{Figure} 7. Langmuir type orbits in the $Ps^{-}$ ion. (a) Shortest
periodic orbit. (b) A more complex one. (c) Poincar\'e map. (d) Detail of
the Poincar\'e map showing the regular structure of the orbits.

\textsc{Figure} 8. Asymmetric top orbits for the $Ps^{-}$ ion. (a) One such
orbit. (b) Symmetry-breaking perturbation on the initial conditions, rapidly
leading to autoionization.

\textsc{Figure} 9. Planetary configurations. (a) $Ps^{-}$ ion: no stable
ones were found. (b) Hypothetical system with charge 2 $e^{+}$ at the
nucleus leads to non-ionizing orbits.

\end{document}
